%----------------------------------------------------------------------------------------
%	PACKAGES AND OTHER DOCUMENT CONFIGURATIONS
%----------------------------------------------------------------------------------------
\documentclass[margin, centered]{res}
\topmargin=-0.5in
\oddsidemargin -.5in
\evensidemargin -.5in
\textwidth=6.5in
\itemsep=0in
\parsep=0in
\newsectionwidth{1in}

\usepackage[pdftex]{graphicx}
\usepackage{enumitem}
\usepackage{wrapfig}
\usepackage{helvet}
\usepackage[colorlinks = true,
			linkcolor = blue,
			urlcolor  = blue,
			citecolor = blue,
			anchorcolor = blue]{hyperref}
			\setlength{\textwidth}{6.5in} % Text width of the document
			\setlength{\textheight}{720pt}
			
			
\begin{document}
%----------------------------------------------------------------------------------------
%	NAME AND ADDRESS SECTION
%----------------------------------------------------------------------------------------\
\begin{center}
	\hspace{-\hoffset}
	\huge\bf{Amanpreet S. Walia}
\end{center}
\vspace{-6mm}
\moveleft\hoffset\vbox{\hrule width 19cm height 0.5pt}
\vspace{-7mm}
\begin{center}
	\hspace{-\hoffset}
	\href{mailto:apsw02@my.yorku.ca}{apsw02@my.yorku.ca} ~\textbullet~ \(+1\) (905)-781-9261 ~\textbullet~ 9,Howell Street, Brampton, ON, Canada
\end{center}
\vspace{-7mm}
\begin{resume}
\section{Education}
\textbf{B.Eng.(Spec. Hons.) Computer Engineering} \hfill 2013 - 2018(Expected) \\
\href{http://lassonde.yorku.ca/}{Lassonde School of Engineering,York University}
\begin{itemize}
	\item GPA of \textbf{8.02}/9 (Dec 2017)
\end{itemize}

%----------------------------------------------------------------------------------------
%	TECHNICAL SKILLS SECTION
%----------------------------------------------------------------------------------------
\section{Technical \hspace{2mm} Skills}
\textbf{Languages \& Frameworks} - C, C++, Java, Python, MATLAB, Eiffel, SQL, Bash, HTML, Verilog, MIPS, OpenCV, GStreamer \\
\textbf{Tools} - Jupyter(IPython), Visual Studio, Qt Creator, Eclipse, Spyder, PyCharm, \LaTeX, Fritzing, PSpice, ModelSim, PTC Creo, MS Visio \\
\textbf{Hardware} - Arduino, Raspberry Pi, Nvidia Jetson TX1, Altera Cyclone II \\
\textbf{Operating Systems} - Linux(Ubuntu), Mac OS, Microsoft Windows
%----------------------------------------------------------------------------------------
%	EXPERIENCE
%----------------------------------------------------------------------------------------
\section{Experience}

\textbf{Research Assistant,  \href{http://cvr.yorku.ca/} { Centre for Vision Research, York University}} 
\hfill September, 2017 - Present \\
Project : \href{http://www.lassondeundergraduateresearch.com/electricalandcomputer#/new-page-s6yt8/}{\emph{Building World's Largest Dynamic Scenes Database}}(Continuation)\\
Responsible for development of software to automate web crawling, cleaning, and annotation.Cleaning of collected videos and annotating them into coherent image regions.
\\\\
\textbf{Research Intern.,  \href{http://cvr.yorku.ca/} { Centre for Vision Research, York University}} 
\hfill May, 2017 - September, 2017 \\
Project : \href{http://www.lassondeundergraduateresearch.com/electricalandcomputer#/new-page-s6yt8/}{\emph{Building World's Largest Dynamic Scenes Database}}\\
Formulated scene categories involving dynamic textures, designed and implemented tools for collecting videos from the web and annotate them using a semi-automated keyframes extraction process to provide accurate ground truths.This new dataset is largest dynamic scenes dataset containing twice the number of categories of the previously existing dataset. 
\\ \\
\textbf{Research Intern.,  \href{http://cvr.yorku.ca/} { Centre for Vision Research, York University}} 
\hfill May, 2016 - September, 2016 \\
Project : \href{http://www.lassondeundergraduateresearch.com/lassonde#/amanpreet-walia/}{\emph{Attentive Sensing for Dynamic Scene Analysis}}\\ 
Developed real-time video acquisition system for network cameras using GStreamer thereby reducing the frame latency by 90\%.Designed \& programmed embedded system to control actuators for pan/tilt camera mirrors over the network using a Raspberry Pi.
\\\\
\textbf{Peer Mentor,  Lassonde School of Engineering, York University}  \hfill September, 2014 - May, 2015 \\
Helped incoming first-year students in transitioning to university life by advising them on upcoming challenges, empowering them to utilize all the services and resources offered by university optimally.
%----------------------------------------------------------------------------------------
%	RELEVANT COURSE SECTION
%----------------------------------------------------------------------------------------
\section{Relevant \hspace{2mm} Courses}
 
Computer Vision, Robotics, Computer Architecture, Digital Communication, Computer Networks, Embedded Systems, Databases, Operating System, Digital Logic \& Design, Software Engineering,Data Structures, UNIX Programming, Electrical Circuits 

%----------------------------------------------------------------------------------------
%	Selected Projects Section
%----------------------------------------------------------------------------------------
\section{Selected Projects}
All projects available on git : \url{https://www.github.com/amanwalia92}\\
%\setlist[itemize]{
\begin{itemize}[leftmargin=*]
    \item \textbf{\href{https://github.com/amanwalia92/EmbeddedEye}{EmbeddedEye}} : Designed an embedded vision system using Nvidia Jetson TX1 for surveillance application.Current system integrates wide FOV camera providing continuous gaze of the larger environment with narrow FOV camera that can be directed to objects of interest using rotating pan/tilt mirrors. \\
    \textit{Our project was among top 5 projects presented at Capstone project conference.} 
	\item \textbf{\href{https://github.com/amanwalia92/VisionChess}{Vision Chess}} : Developed real-time python computer vision application to digitally model chess moves made by players using a standard web cam.Our System successfully recorded a complete game of chess.
	\item \textbf{\href{https://github.com/amanwalia92/EmbeddedNotifier}{Embedded Notifier}} : An embedded system project to display social media notifications on 16$\times$2 LCD module using Raspberry  Pi.Successful in reading unread notifications from Facebook and Gmail accounts using respective Python SDKs.
	\item \textbf{\href{https://github.com/amanwalia92/Multi-Threaded-Chat-Network}{P2P Chat Application}} : Developed distributed messaging system in Java containing a hybrid architecture of central server for user registration and Peer-to-peer communication for exchanging messages between two users.
	\item \textbf{\href{https://github.com/amanwalia92/Multi-Threaded-Chat-Network}{Customer Portal}} : Developed JDBC application using JavaFX GUI toolkit to interact with Database retrieving and updating records.
\end{itemize}
%----------------------------------------------------------------------------------------
%	ACHIEVEMENT SECTION
%----------------------------------------------------------------------------------------

\section{Awards \& Honours}
\begin{itemize}[leftmargin=*]
	\item  \href{http://www.lassondeundergraduateresearch.com/electricalandcomputer#/new-page-s6yt8/}{\emph{NSERC Undergraduate Student Research Award}} in 2017.
	\item  \href{http://www.lassondeundergraduateresearch.com/lassonde#/amanpreet-walia/}{\emph{NSERC Undergraduate Student Research Award}} in 2016.
	\item
	H. Marie Scholarship for Excellence in Engineering for having the highest grade point average in 2016.
	\item
	Enbridge Engineering Scholarship in 2016.
	\item
	York University Continuing Student Scholarship in 2015.
	\item
	The Gordon and Agnes (Twambley) Brash Award in 2015.
	\item
	\href{http://physics.info.yorku.ca/undergraduate-studies/undergraduate-awards/departmental-awards/#a2}{\emph{Denise Hobbins Prize for highest marks in Physics Course}} in 2014.
	\item
	York University Continuing Student Scholarship in 2014.
	
	
\end{itemize}

\end{resume}






\end{document}